\documentclass{article}
\usepackage{graphicx}
\usepackage{hyperref}
\usepackage{changepage}

\begin{document}

\begin{center}
    {
    \Large
    Trabalho Prático I - Geometria Computacional

    DCC207 - Algoritmos II
    
    }

    Departamento de Ciência da Computação, Instituto de Ciências Exatas, Universidade Federal de Minas Gerais

    Belo Horizonte, Minas Gerais

    Outono de 2025
\end{center}

\begin{flushright}
2022101086  Augusto Guerra de Lima\\
\href{mailto:augustoguerra@dcc.ufmg.br}{\texttt{augustoguerra@dcc.ufmg.br}}\\
2023028234  Cauã Magalhães Pereira\\
\href{mailto:cauamp11@gmail.com}{\texttt{cauamp11@gmail.com}} \\
xxxxxxxxxx  Heitor Gonçalves Leite\\
\href{mailto:}{\texttt{}}
\end{flushright}

\hrule
\vspace{.1cm}
\hrule
\begin{adjustwidth}{-1.5cm}{-1.5cm}

\section{Introdução}
\ 

Este trabalho objetiva estudar como estruturas de dados associadas a geometria computacional, a saber, árvores \(k\)-dimensionais, são empregadas no contexto de georreferenciamento.

\section{Referências}

\end{adjustwidth}
\end{document}